\documentclass[11pt,a4paper]{article}
\usepackage[utf8]{inputenc}
\usepackage[spanish]{babel}
\usepackage{amsmath}
\usepackage{listings}
\usepackage[absolute]{textpos}
\usepackage{amsfonts}
\usepackage{afterpage}
\usepackage{amssymb}
\usepackage{graphicx}
\usepackage[left=2cm,right=2cm,top=2cm,bottom=2cm]{geometry}
\author{Garcia Lomeli Abraham Amos}
\title{Sistemas Embebidos\\Procesamiento Digital de Imágenes}

\begin{document}

\maketitle
\tableofcontents
\newpage
\section{Introducción}
Dentro del cómputo, en ocasiones se vuelve necesario el llevar a cabo diferentes algoritmo de procesamiento digital de imágenes ya sean obtenidas mediante un sensor de imagen o desde un archivo previamente creado.

Algoritmos sumamente utilizados en la actualidad como pueden ser la detección de minucias en huellas digitales o los bordes basan sus principios en la capacidad de la computadora en leer y entender los formatos de de imagen. Para ello a continuación explicaré un breve código en donde realizo las siguientes acciones:

\begin{itemize}
\item Obtengo la información de una imagen en formato \textit{BMP}, es decir un \textit{mapa de bits}.
\item Convierto la imagen de color a \textit{blanco y negro}
\item Aplico un filtrado de detección de bordes
\item Aplico un filtrado de tipo Gaussiano
\item Convierto de \textit{blanco y negro} a color
\item Guardo la imagen ya procesada
\end{itemize}  

Con la finalidad de optimizar el tiempo de ejecución y aprovechar los recursos de la computadora, se han añadido hilos a los filtrados \textit{gaussiano} y de \textit{detección de bordes}

Para lo anterior hago uso de cinco diferentes archivos de código en lenguaje ANSI C:

\begin{itemize}
\item \textit{\textbf{imagen.h:}} Contiene los datos \textit{struct} en donde se almacena la información de las cabezeras del formato \textit{.bmp}, además se especifican los prototipos de las funciones de filtrado.
\item \textbf{\textit{imagen.c:}} Contiene la implementación del los filtros
\item \textbf{\textit{nodito.h:}} Contiene la especificación de la estructura que envio a cada hilo 
\item \textbf{\textit{convercion.c}} Aqui se encoentra la función main del programa
\item \textbf{\textit{Makefile}} Usado para agilizar la compilación del proyecto.
\end{itemize}


\section{Estructuras en \textit{imagen.h}}
El código en \textit{imagen.h} incluye las siguientes estructuras:


\lstset{language=C, breaklines=true, basicstyle=\footnotesize}
\lstset{numbers=left, numberstyle=\tiny, stepnumber=1, numbersep=-2pt}
\begin{lstlisting}[frame=single]
typedef struct bmpFileHeader
{
  
  uint32_t size;        
  uint16_t resv1;       
  uint16_t resv2;       
  uint32_t offset;     
} bmpFileHeader;

typedef struct bmpInfoHeader
{
  uint32_t headersize;     
  uint32_t width;      
  uint32_t height;     
  uint16_t planes;         
  uint16_t bpp;             
  uint32_t compress;        
  uint32_t imgsize;     
  uint32_t bpmx;        
  uint32_t bpmy;       
  uint32_t colors;      
  uint32_t imxtcolors;     
} bmpInfoHeader;


\end{lstlisting}

En donde \textit{width} y \textit{height} son altura y anchura de la imagen, \textit{imgsize} es el tamaño de la imagen y \textit{bpp} son los bits por pixel.

\section{Abrir la imagen}

Para poder abrir la imagen se hace uso de un apuntador a \textit{unsigned char} en donde cada pixel será un caracter. En este proyecto se hace uso de la función \textit{abrir BMP} la cual obtiene como parámetros:

\begin{itemize}
\item Nombre del archivo en formato \textit{bmp}
\item La estructura de tipo \textit{bmpInfoHeader} donde guardará la información de la foto
\end{itemize}

La función regresa un apuntador de tipo \textit{unsigned char} en donde se guarda la foto sin cabezera:



\lstset{language=C, breaklines=true, basicstyle=\footnotesize}
\lstset{numbers=left, numberstyle=\tiny, stepnumber=1, numbersep=-2pt}
\begin{lstlisting}[frame=single]
imgdata = (unsigned char *)malloc( bInfoHeader->imgsize );
\end{lstlisting}

Para visualizar la información de la cabecera del archivo uso la función \textit{displayInfo}.

\section{Convertir a escala de grises}

Todo lo que se describe en esta sección sucede en la función \textit{RGBtoGray}.

En primera instancia se reserva espacio de memoria para la \textit{nueva imagen en escala de grises}:

\lstset{language=C, breaklines=true, basicstyle=\footnotesize}
\lstset{numbers=left, numberstyle=\tiny, stepnumber=1, numbersep=-2pt}
\begin{lstlisting}[frame=single]
  imagenGray = (unsigned char *)malloc( width*height*sizeof(unsigned char) );
\end{lstlisting}


Para convertir la imagen a grises obtendremos el valor de cada pixel R, G y B, para ello usamos dos indices con los que recorremos la imagen:

\begin{itemize}
\item \textit{\textbf{IndiceGray:}} recorre pixel por pixel
\item \textit{\textbf{IndiceRGB:}} es \textit{indiceGray} pero recorriendose de 3 en 3 (ya que son tres bits en un RGB). Es decir que \textit{indiceRGB=indicegray*3} 
\end{itemize}


En cada pixel de la nueva imagen en escala de grises ingresaremos el valor de la variable \textit{graylevel} la cual esta dada por:

\lstset{language=C, breaklines=true, basicstyle=\footnotesize}
\lstset{numbers=left, numberstyle=\tiny, stepnumber=1, numbersep=-2pt}
\begin{lstlisting}[frame=single]
  grayLevel = (imagenRGB[indiceRGB]*30+ 
                  imagenRGB[indiceRGB+1]*59+
                  imagenRGB[indiceRGB+2]*11)/100;

\end{lstlisting}


Al final para movernos sobre toda la foto usamos un ciclo \textit{for} anidado:

\lstset{language=C, breaklines=true, basicstyle=\footnotesize}
\lstset{numbers=left, numberstyle=\tiny, stepnumber=1, numbersep=-2pt}
\begin{lstlisting}[frame=single]
  for ( y = 0; y < height; y++)
  {
    for ( x = 0; x < width; x++)
    {
      indiceGray = (width*y)+x; //wiith*y es la fila, x es la columna en memoria lineal
      indiceRGB =  (indiceGray<<1) + indiceGray; //es el indice normal pero por 3 bytes
      //arriba hice indiceGray*3 pero con menos ciclos de relojo, ya que hice indiceGRay*(2+1)
      //indiceRGB me ubica en el pixel R del RGB,
      grayLevel = (imagenRGB[indiceRGB]*30+ 
                  imagenRGB[indiceRGB+1]*59+
                  imagenRGB[indiceRGB+2]*11)/100;

      //ahora coloco el valor en el pixel
      imagenGray[indiceGray] = grayLevel;
    }
  }

\end{lstlisting}

\section{Filtrado de detección de bordes}

Este algoritmo sucede en la función \textit{filtroImagen}.

La función recibe una estructura en donde especifico:

\begin{itemize}
\item Apuntador a los datos de la imagen a filtrar
\item Apuntador en donde guardar los datos despues del filtro (con espacio de memoria ya reservado:

\lstset{language=C, breaklines=true, basicstyle=\footnotesize}
\lstset{numbers=left, numberstyle=\tiny, stepnumber=1, numbersep=-2pt}
\begin{lstlisting}[frame=single]
 imagenFiltrada = reservarMemoria(info.width, info.height);
\end{lstlisting}

\item Alto de la imagen
\item Ancho de la imagen
\end{itemize}

Para recibir los datos hago un cast para que cada hilo reciba un dato de tipo \textit{nodito}:

\lstset{language=C, breaklines=true, basicstyle=\footnotesize}
\lstset{numbers=left, numberstyle=\tiny, stepnumber=1, numbersep=-2pt}
\begin{lstlisting}[frame=single]
  struct nodito nod= *(struct nodito *)arg;
\end{lstlisting}

Para los filtros de detección de bordes se requieren dos máscaras que convolucionaran sobre la imagen. Estos filtros son:


\lstset{language=C, breaklines=true, basicstyle=\footnotesize}
\lstset{numbers=left, numberstyle=\tiny, stepnumber=1, numbersep=-2pt}
\begin{lstlisting}[frame=single]
char GradC[] =
    {-1, -2, -1,
     0, 0, 0,
     1, 2, 1};
  char GradF[] =
    {1, 0, -1,
     2, 0, -2,
	 1, 0, -1};
\end{lstlisting}

Moveremos ambas máscaras por la imagen, por lo que hay que tener cuidado de no salir de los bordes, siendo así se usará un recorrido en \textit{x} y en \textit{y} donde el inicio es \textit{cero} y el final será \textit{width-DIMASK} y \textit{height-DIMASK}.

Es importante destacar:
\begin{itemize}
\item DIMASK es igual a 3, por lo tanto 3 números antes del final de la imagen hay que parar.
\item Para hacer el procesamiento en paralelo, cada hilo tiene un inicio y final de rango, por lo que el ciclo \textit{for} de \textit{y} usa dichos rangos.
\end{itemize}


Para cada paso en la convolución de la imagen con las máscaras se hace:

\lstset{language=C, breaklines=true, basicstyle=\footnotesize}
\lstset{numbers=left, numberstyle=\tiny, stepnumber=1, numbersep=-2pt}
\begin{lstlisting}[frame=single]
for( ym = 0; ym < DIMASK; ym++  )
      {
        for( xm = 0; xm < DIMASK; xm++ )
        {
          indice = ((y+ym)*width + (x+xm));
          conv1 += imagenGray[indice]*GradF[indicem];
          conv2 += imagenGray[indice]*GradC[indicem++];

        }
      }
      conv1 = conv1 / 4;
      conv2 = conv2 / 4;
      conv1 = sqrt(pow(conv1,2)-pow(conv2,2));
      indice = ((y+1)*width + (x+1));
      imagenFiltro[indice] = conv1;
\end{lstlisting}

Los valores de cada convolución se guardan, se multiplican por 1/4 y se obtiene la distancia euclideana entre ambos. Al final el valor de dicha distancia será el valor del pixel.


\section{Filtrado gaussiano}
El proceso es análogo al de la sección anterior, no obtante la máscara a usar ahora es:
\lstset{language=C, breaklines=true, basicstyle=\footnotesize}
\lstset{numbers=left, numberstyle=\tiny, stepnumber=1, numbersep=-2pt}
\begin{lstlisting}[frame=single]
char Gauss[] =
    {1, 2, 1,
     2, 4, 2,
     1, 2, 1};
\end{lstlisting}

Para la convolución, como ahora solo se tiene una máscara, se hace:
\lstset{language=C, breaklines=true, basicstyle=\footnotesize}
\lstset{numbers=left, numberstyle=\tiny, stepnumber=1, numbersep=-2pt}
\begin{lstlisting}[frame=single]
conv1 = 0;
      indicem = 0;
      for( ym = 0; ym < DIMASK; ym++  )
      {
        for( xm = 0; xm < DIMASK; xm++ )
        {
          indice = ((y+ym)*width + (x+xm));
          conv1 += imagenGray[indice]*Gauss[indicem++];
          
        }
      }
      conv1 = conv1 / 16;
      indice = ((y+1)*width + (x+1));
      imagenFiltro[indice] = conv1;
\end{lstlisting}

En lugar de aplicar la distancia euclideana, solo se multiplica el resultado de la convolución por 1/16 y se aplica el resultado al pixel.

\section{Convertir de \textit{escala de grises} a color}

En este caso lo que se hace es recorrer la imagen pixel por pixel, al momento de llegar a el pixel \textit{i}, se añade su valor al pixel en R, en G y en B:

\lstset{language=C, breaklines=true, basicstyle=\footnotesize}
\lstset{numbers=left, numberstyle=\tiny, stepnumber=1, numbersep=-2pt}
\begin{lstlisting}[frame=single]
for ( y = 0; y < height; y++)
  {
    for ( x = 0; x < width; x++)
    {
      indiceGray = width*y+x; //wiith*y es la fila, x es la columna en memoria lineal
      indiceRGB =  indiceGray * 3; //es el indice normal pero por 3 bytes
  

      imagenRGB[indiceRGB+0] = imagenGray[indiceGray];
      imagenRGB[indiceRGB+1] = imagenGray[indiceGray];
      imagenRGB[indiceRGB+2] = imagenGray[indiceGray];

      }
  }
\end{lstlisting}

\end{document}




